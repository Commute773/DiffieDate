\documentclass[12pt]{article}
\usepackage{amsmath, amssymb}
\usepackage{hyperref}
\usepackage{graphicx}
\usepackage[margin=1in]{geometry}
\usepackage{listings}
\usepackage{color}
\usepackage{mathtools}
\usepackage{physics}

\title{DiffieDate: Private Mutual Matching via Garbled Circuits}
\author{}
\date{}

\begin{document}

\maketitle

\section*{Overview}

Garble Date is a demo application implementing a private mutual matching protocol based on garbled circuits and elliptic curve Diffie-Hellman (ECDH). It allows users to express preferences across various categories (e.g., ``Date'', ``Fight'', ``Have child with'') without revealing any single-sided preferences unless a mutual match occurs.

The protocol is designed such that:
\begin{itemize}
  \item No third party (including the server) learns individual preferences.
  \item Mutual preferences can be confirmed without revealing non-matching ones.
  \item Cryptographic keys are derived from ECDH key exchange, ensuring each user pair has a unique shared secret.
\end{itemize}

\vspace{1em}

\section*{Mathematical Foundations}

\subsection*{ECDH Key Exchange}

Each user has a persistent elliptic curve key pair for each category:
\[
\text{User } U_i \rightarrow (sk_i, pk_i)
\]
Public keys are registered on the server. Private keys are stored locally.

To derive a shared secret between users $A$ and $B$:
\[
S_{AB} = \text{ECDH}(sk_A, pk_B) = sk_A \cdot pk_B = sk_B \cdot pk_A
\]

We strip the leading byte from the compressed shared secret (as required by the noble library used in implementation).

\subsection*{HKDF Derivation of Wire Labels}

Using the shared secret $S$, we derive the labels for Bob's input wire and the output wire:

\[
\begin{aligned}
B_0 &= \text{HKDF}_{\text{SHA-256}}(S, \texttt{``B0''}) \\
B_1 &= \text{HKDF}_{\text{SHA-256}}(S, \texttt{``B1''}) \\
R_0 &= \text{HKDF}_{\text{SHA-256}}(S, \texttt{``R0''}) \\
R_1 &= \text{HKDF}_{\text{SHA-256}}(S, \texttt{``R1''})
\end{aligned}
\]

Alice generates fresh random labels $A_0$, $A_1$ for her input wire.

\subsection*{Garbled Circuit Construction}

We construct a single AND gate. The truth table is:

\[
\begin{array}{ccc}
A & B & R = A \land B \\
\hline
0 & 0 & 0 \\
0 & 1 & 0 \\
1 & 0 & 0 \\
1 & 1 & 1 \\
\end{array}
\]

For each input pair $(A_i, B_j)$, we derive a symmetric key using XOR:
\[
K_{ij} = A_i \oplus B_j
\]

Each output label $R_k \in \{R_0, R_1\}$ is encrypted using AES-GCM under $K_{ij}$ with a random IV:
\[
C_{ij} = \text{AES-GCM}_{K_{ij}}(R_k)
\]

The full garbled table is:
\[
\text{table} = \{ C_{00}, C_{01}, C_{10}, C_{11} \}
\]

The table is randomly shuffled to hide row ordering.

\subsection*{Protocol Steps}

\subsubsection*{Step 1: Registration}

Each user generates a key pair $(sk, pk)$ per category and registers with:
\[
\text{user ID} = \text{SHA-256}(pk) \concat \text{username} \concat \text{category}
\]

\subsubsection*{Step 2: Alice Likes Bob}

Alice derives shared secret $S$, computes $B_0$, $B_1$, $R_0$, $R_1$ via HKDF.

She generates a garbled AND gate using fresh $A_0$, $A_1$ and the derived $B_i$, $R_i$.

She chooses her input label:
\[
a = \text{bit indicating like or not} \in \{0,1\},\quad A = A_a
\]

Alice sends to the server:
\[
\text{GarbledLike} = \{ \text{from}, \text{to}, \text{garbledTable}, A, (R_0, R_1) \}
\]

\subsubsection*{Step 3: Bob Responds}

Bob derives $S$, then derives $B_0$, $B_1$ using HKDF. He picks his label $B = B_b$ and sends it to the server.

\subsubsection*{Step 4: Server Evaluation}

The server cannot decrypt anything. It simply relays the data.

\subsubsection*{Step 5: Client Decryption}

The client (either Alice or Bob) uses:
\[
K = A \oplus B
\]
to try decrypting each ciphertext in the shuffled table until one succeeds. The result is a label $R$.

If $R = R_1$, the output is 1 — a match. Otherwise, not.

\[
\text{Match} \iff R = R_1
\]

\section*{Privacy Guarantees}

\begin{itemize}
  \item \textbf{Server blindness:} The server never learns $a$ or $b$, nor the result.
  \item \textbf{Unilateral secrecy:} If only Alice likes Bob, Bob never learns.
  \item \textbf{Determinism:} Given keys are persistent, labels are always the same for a given user-pair.
  \item \textbf{Forward secrecy:} Assuming ephemeral keys are used (optional), mutual likes can't be retroactively decrypted.
\end{itemize}

\section*{Implementation Notes}

\begin{itemize}
  \item AES-GCM is used with 128-bit keys and 12-byte IVs.
  \item XOR is used as a toy key-derivation; replace with a secure KDF like HKDF in production.
  \item All messages are serialized in base64 and stored as strings.
  \item Client-side logic performs decryption; server acts as a blind relay.
\end{itemize}

\end{document}
